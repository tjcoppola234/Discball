\documentclass[10pt]{article}

\usepackage[a4paper, left=2cm, right=2cm]{geometry} % A4 paper size and thin margins
\usepackage{xcolor} % Required for specifying custom colours
\definecolor{grey}{rgb}{0.9,0.9,0.9} % Colour of the box surrounding the title
\usepackage[utf8]{inputenc} % Required for inputting international characters
\usepackage[T1]{fontenc} % Output font encoding for international characters
\usepackage[sfdefault]{ClearSans} % Use the Clear Sans font (sans serif)
%\usepackage{XCharter} % Use the XCharter font (serif)
\usepackage{graphicx}
\graphicspath{ {../assets/} }

\usepackage{amsmath}
\usepackage{geometry}

\usepackage{hyperref}
\hypersetup{
    colorlinks=true,
    linkcolor=blue,
    filecolor=magenta,      
    urlcolor=blue,
}

\title{Discball Official Rulebook}
\author{Theo Coppola}

\begin{document}
    % Title page template by: Frits Wenneker and Vel (vel@latextemplates.com)
    \begin{titlepage} % Suppresses displaying the page number on the title page and the subsequent page counts as page 1

        %	Grey title box
        \colorbox{grey}{
            \parbox[t]{0.93\textwidth}{ % Outer full width box
                \parbox[t]{0.91\textwidth}{ % Inner box for inner right text margin
                    \raggedleft% Right align the text
                    \fontsize{50pt}{80pt}\selectfont % Title font size, the first argument is the font size and the second is the line spacing, adjust depending on title length
                    \vspace{0.7cm} % Space between the start of the title and the top of the grey box
                    
                    Discball:\\
                    Official Rulebook\\
                    
                    \vspace{0.7cm} % Space between the end of the title and the bottom of the grey box
                }
            }
        }

        \vfill
        
        \begin{center}
            \includegraphics{title_image}
        \end{center}
        
        \vfill % Space between the title box and author information
        
        %	Author name and information
        \parbox[t]{0.93\textwidth}{ % Box to inset this section slightly
            \raggedleft% Right align the text
            \large% Increase the font size
            {\Large Theo Coppola}\\[4pt] % Extra space after name
            Stamford, CT\\[4pt] % Extra space before URL
            \texttt{\href{https://github.com/tjcoppola234/Discball}{GitHub}}\\
            
            \hfill\rule{0.2\linewidth}{1pt}% Horizontal line, first argument width, second thickness
        }
        
    \end{titlepage}

    \renewcommand*\contentsname{Table of Contents}
    \tableofcontents

    \newpage

    \section{Introduction}

    Birthed by recent high school grads in the summer of covid (2020) this beautiful game is a fast and fun mixup of some of the highest velocity sports out there\ldots and it only exists because the lacrosse goals were left unlocked on the field. Turn to discball instead of ultimate frisbee when: 
    \begin{itemize}
        \item You don't have enough people for a full game
        \item You only have access to half of a football field
        \item You want to show people how accurate of a thrower you are
        \item You want to try something new
        \item Lacrosse goals are available
    \end{itemize}
    The game plays quick and you will most likely get gassed, but it will be worth it. It's a great way to learn how to curve your throws, throw at different heights, and work as a team.

    \section{The Field}

    \includegraphics[width=.7\textwidth]{field/field}

    \vspace{5pt}

    Field Terms:
    \begin{itemize}
        \item \textbf{Crease}: The circle at the end of the field
        \item \textbf{Goal}: The lacrosse goal inside the crease, represented by a small rectangle.
        \item \textbf{Inner Box}: The rectangular space on the immediate front side of the crease.
        \item \textbf{Outer Box}: The area surrounding the Inner Box, but before the half-field line.
        \item \textbf{The Range}: The space past the half-field line. This area is the other side of the field for a given team.
    \end{itemize}

    \section{Starting the game}
    \section{Playing the game}
    \section{Scoring}

    This image shows how points are awarded when a team is aiming for the goal on the left side of the field.

    \vspace{5pt}

    \includegraphics[width=.7\textwidth]{field/field_points}

    \begin{itemize}
        \item 
    \end{itemize}

    \section{Penalties}

    Players call the penalties. If a player thinks a foul occured, they call it out and play stops. If the offending player agrees that a foul occured, then play resets and 
    \begin{enumerate}
        \item If the fouled player was handling the disc, they retain possession at the their current position.
        \item If the fouled player was receiving a pass, the original handler retains possession at the point of the throw.
        \item If the fouled player was defending against a pass, a turnover occurs at the position where the foul occured.
        \item If the fouled player was marking the handler, a turnover occurs at that position.
    \end{enumerate}

    \subsection{Other penalties}
    \begin{itemize}
        \item Defensive and offensive players are not allowed in the crease. 
        \begin{itemize}
            \item If an offensive player steps into the crease, a turnover occurs at the position of the handler.
            \item If a defensive player steps into the crease, they must go behind the goal and do a pushup, then they can return to play.
        \end{itemize}
        Players are allowed to jump across the crease, but they must quickly leave the crease if they land in it.
        \item If all but one defensive players are not behind the half-field line on the side of the field opposite the offense when the disc is checked, the offensive checker retains possession of the disc at the middle of the field, in front of the half-field line.
    \end{itemize}

\end{document}